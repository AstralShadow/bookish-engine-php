\documentclass[12pt]{article}
%\usepackage[a4paper,  margin=.8in]{geometry}
\usepackage[a4paper, margin=1.5in]{geometry}

% \subimport{relative path}{file}
\usepackage{import}

% Links
\usepackage{hyperref}

% Images
\usepackage{graphicx}
\usepackage{pdfpages}

% Date
\usepackage{datetime}

% Code snippets
\usepackage{listings}

% Chapter title format
\usepackage{titlesec}
\titleformat{\chapter}[block]
    {\normalfont\huge\bfseries}{\thechapter.}{1em}{\Huge}
\titlespacing*{\chapter}{0pt}{-19pt}{0pt}

% Language
\usepackage{polyglossia}
\setdefaultlanguage{bulgarian}
\setotherlanguages{english}
\PolyglossiaSetup{bulgarian}{indentfirst=true}

%\usepackage[bulgarian]{babel}
%\setmainfont{FreeSerif}
%\setsansfont{FreeSans}
%\setmainfont{LinuxLibertine}
\setmainfont{Times New Roman}


\newif\ifExtra
%\Extratrue % Comment this to remove most reference links

\newif\ifnoit
\noittrue % Uncomment to get noit version. Don't run with extra.


\input{private/title}

\begin{document}

%\maketitle
%\tableofcontents

\newgeometry{margin=1in}

\begin{center}
    {\huge Проект \appId{}}
\end{center}

\section{Тема: \appTitle{}}
\subimport{private/}{team}

\section{Резюме}
\subsection{Цели}
Проектът цели да насърчи и улесни изучаването на
големи количества информация. Потребителите качват
синтезирани и разбираеми ресурси, като в замяна
получават шанса да свалят други такива.

\subsection{Основни етапи в реализирането на проекта}
\begin{itemize}
    \item Създаване на фреймуърк
    \item Измисляне на идея за проект
    \item Планиране на архитектурата
    \item Имплементация на интерфейса и контролерите
    \item Тестване и подобряване
\end{itemize}

\subsection{Ниво на сложност на проекта}
\begin{itemize}
    \item Създаване на функционален фреймуърк
    \item Имплементация на MVC, което позволява
        преизползване на контролери
    \item Създаване на ORM функционалност, кеширане
        и оптимизиране на заявки до база данни
    \item Създаване и използване на специализирана
        система за динамизация на уеб-интерфейс
    \item Обработване и филтриране на HTTP заявки
    \item Създаване на RESTful API.
    \item Имплементация на стандартни практики
        за кибер сигурност
\end{itemize}

\subsection{Логическо и функционално описание на решението}
\subimport{SweetTalk}{intro}

\subsection{Реализация}
\subimport{RealWorkCopy/}{main}

%\subimport{Intro/}{main}
%\subimport{private/}{team}
%\subimport{SweetTalk/}{main}
%\subimport{RealWork/}{main}
%\subimport{Extra/}{main}

\subsection{Описание на приложението}
Приложението изисква потребителска регистрация за
повечето функционалности. Тя може да бъде направена
в уеб браузър чрез страницата за регистрация или
чрез програмното API, което позволява възможността за
създаване на различни потребителски интерфейси.

Всеки потребител може да качва своите ресурси,
специален екип от модератори ги одобрява, и
всеки потребител може да замени свитъците, които е
получил за своите ресурси, с ресурси, качени от
други потребители на приложението.

Приложението може да бъде достъпено на следващите адреси:

\subsubsection{Уеб адрес}
\input{Extra/private/uri}

\subsubsection{Сорс код}
\href
    {https://github.com/AstralShadow/bookish-engine-php}
    {https://github.com/AstralShadow/bookish-engine-php}

\subsection{Заключение}
\subimport{Intro/}{main}


\end{document}
