\subsubsection{API интерфейс}
Това са контролерите, които отговарят за програмния
интерфейс на приложението. Те се намират в директорията
API и имената им съответстват на адресите, на които могат
да бъдат достъпени. При употребата им потребителската
сесия се предава като част от съдържанието на заявката,
което премахва необходимостта от бисквитки и CSRF токени.

Те следват съвременния RESTful стандарт с цел улеснение
на употребата им от други приложения, като настолни или
мобилни клиенти.

Основната функционалност на приложението се изпълнява
от тези контролери.

\begin{itemize}

    \item{Потребител}
    \begin{itemize}
        \item GET /api/user
            Предоставя информация за настоящия профил
        \item POST /api/user
            Позволява регистрация на нов потребител
        \item POST /api/user/avatar
            Позволява смяна на потребителскя аватар
            (профилно изображение)
        \item GET /api/user/<name>
            Предоставя публична информация (име, аватар)
        \item GET /api/user/<name>/avatar
            Предоставя потребителски аватар
        \item DELETE /api/user
            Изтриване на потребителски профил
    \end{itemize}

    \item{Сесия}
    \begin{itemize}
        \item GET /api/session
            Информация за сесията
        \item POST /api/session
            Създаване на сесия.
            Изисква потребителско име и парола
        \item DELETE /api/session
            Изтриване на сесия. Също известно като
            излизане от потребителски акаунт.
    \end{itemize}

    \item{Ресурс}
    \begin{itemize}
        \item POST /api/resource
            Създаване на ресурс
        \item PUT /api/resource/<id>
            Редакция на ресурс
        \item DELETE /api/resource/<id>
            Изтриване на ресурс
        \item GET /api/resource/<id>
            Достъпване на информация за ресурс
        \item GET /api/resource/<id>/preview
            Изтегляне на демонстративен материал
        \item POST /api/resource/<id>/buy
            Закупуване на ресурс.
        \item GET /api/resource/<id>/download
            Изтегляне на ресурс.
    \end{itemize}

    \item{Ключови думи}
    \begin{itemize}
        \item GET /api/tags
            Списък с ключови думи
        \item POST /api/tags
            Създаване на ключова дума
    \end{itemize}

    \item{Търсене и Филтриране}
    \begin{itemize}
        \item /api/search/<words>
        \item /api/filter/<keywords>
    \end{itemize}
\end{itemize}

Търсените елементи се разделят със знак плюс +.
Филтрирането е по-бърза операция, докато търсенето
позволява въвеждане на непълни ключови думи.

\subsubsection{Уеб интерфейс}
Теси контролери се намират в директорията Controllers
и са разделени спрямо логическите действия, за които
отговарят. Те действат предимно като потребителски
интерфейс, като при нужда извикват съответните контролери
от програмния интерфейс.
Всеки от тези контролери използва един или повече от
шаблоните в директорията Templates, които следват
специален синтакс за въвеждане на конкретни данни в тях.

\begin{itemize}
    \item Home
        Началната страница
    \item Account
        Страниците за вход, регистрация и изход
    \item User
        Потребителска страница, създаване на ресурс и
        смяна на потребителски аватар
    \item Resource
        Преглед, закупуване, изтегляне и оценяване на
        ресурс.
    \item Search
        Търсене и филтриране на ресурси.
\end{itemize}

