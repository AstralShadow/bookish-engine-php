%\chapter{Описание и цел}

Изготвянето и намирането на качествени учебни материали
е сложна и времеемка процедура, чиито резултати са крайно
необходими за постигане на качествено обучение и 
самообучение в училищните и университетските среди.

Съвременната образователна система полага усилия за
улесняването ѝ чрез изготвяне на учебни програми и
издаване на специализирани учебници, които се използват
в българските училища. Но за съжаление информацията в
тези учебни пособия често остарява докато дойде моментът
учениците и студентите да учат по тях.

Ние живеем в ера на постоянни открития и изобретения,
всяко от които имащо потенциала да измени учебниците
до основи. Най-потърпевша от този факт област на
образованието е тази на висшите учебни заведения.
В тях се разглеждат обширно конкретни дялове от
съвременните науки, а почти всяко откритие в тях води
до драстично изменение на изучаваните ресурси.
Поради обема на тези ресурси за студентите често е
невъзможно да обхванат цялата информация в рамките на
учебния семестър.


Приложението "\appTitle{}" има за цел да реши точно този
проблем, като предостави на студентите начин ефикасно
да синтезират огромното количество от информация, която
трябва да обхванат в рамките на учебния семестър.

Системата се основава на равностойна обмяна на съкратени
и лесни за разбиране учебни материали. Контрол на
качеството на материалите ще се извършва от специално
подбран екип от доверени потребители, както и чрез
възможността на всеки, който е получил достъп до даден
ресурс да постави публичена оценка и мнение.
Допълнително ще се изисква предоставяне на демонстративна
версия на всеки учебен материал, който да бъде използван
за да се потвърди тематиката и съдържанието на желания
ресурс преди вземането му.


Разработените учебни теми биват предоставени от самите
студенти, като при прочитането им от другите потребители
на приложението авторът получава т.н. Свитъци,
използвани като разменна валута в рамките
на приложението. Тези свитъци могат да бъдат използвани
за придобиване на достъп до други учебни ресурси, което
прави възможна равностойната размяна.

