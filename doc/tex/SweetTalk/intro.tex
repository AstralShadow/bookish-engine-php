Приложението включва специализиран фреймуърк, който има
за цел да позволи лесна разработка на RESTful MVC уеб
приложения. 

Проектът бива разделен на няколко ключови елемента:
\begin{itemize}
    \item Ядро на фреймуърка. Съдържа имплементацията
        на базовите класове за заявки и модели, както и
        основната логика зад рутирането на заявките и
        обработването на различните типове отговори.
    \item Слой за данни. Всеки тип данни бива представян
        от специализиран клас. Всички елементи в
        приложението достъпват базата данни само
        чрез този слой.
    \item Контролери. Обработват заявките и връщат
        отговори. Разделени са в две логически категории -
        контролери, отговарящи за потребителския
        интерфейс и контролери, обработващи заявки до
        програмния интерфейс на приложението.
    \item Рутираща система. Обработва адреса на заявката
        и избира най-подходящия контролер за нея.
    \item Шаблони. Използват се от контролерите, които
        отговарят за потребителския интерфейс. Те са
        специализирани уеб страници, в които контролерите
        могат да вмъкват стойнисти на конкретни места.
    \item Разширения. Съдържа серия от полезни фунции и
        класове, които не са част от ядрото на фреймуърка,
        защото изпълняват задачи, специализирани към
        конкретното приложение.
\end{itemize}

